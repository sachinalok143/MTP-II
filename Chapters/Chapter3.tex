% Chapter Template

\chapter{Literature Survey} % Main chapter title

\label{Chapter2} % Change X to a consecutive number; for referencing this chapter elsewhere, use \ref{ChapterX}

\lhead{Chapter 3. \emph{Literature Survey}} % Change X to a consecutive number; this is for the header on each page - perhaps a shortened title
Human mind is not static; it fluctuates over time especially in an attention demanding task. When our brain fluctuates, our mind works in both task-relevant and irrelevant processes but in a less detailed manner.\cite{grandchamp2014oculometric}, We have worked on detecting wandering of our mind using EEG signal in which electrical activity of the brain has been recorded using an electro physiological method. We have also observed the brain waves in EEG signal in the frequency domain. Among various classifiers we have used J48 algorithm to generate a decision tree from our featured data. For classification we have used SVM classifier. By giving a set of trained data example SVM built a model that assigned our data’s in positive or negative MW categories. 

Few researchers also have worked on this field. Julia W. Y. Kam Julia et al. \cite{kam2013mind} have used two experiments, in one experiment they have used traditional measures of performance  and found that both automatic and volitional and  forms of visual–spatial consideration arranging were fundamentally constricted when MW episodes occurred. In the second experiment ERPs have used to examine whether time frames wherein we momentarily weaken the preparing of outer improvement contributions as our considerations float away from the on-going main job or cortical hypersensitivities in headaches reach out to MW.Another group Romain Grandchamp et al. \cite{grandchamp2014oculometric} worked on oculometric variations during MW. They have worked on blink frequency, pupil size and gaze position to check if they were correlated with the occurrence and time course of self-reported MW episodes. 

Another group Jaechoon et al have worked on online education limitations by performing a verification study to see if  mind wandering can be detected using high frequency words to resolve limitations that currently exists. They established a Minimum Learning Judgment System (MLJS) in which vido-based online lecture has been used to detect Mind wandering. 

One more group Benjamin W. Mooney- hamet al.have worked on Mind states by analysising the Neural Bases of alertness and Mind wandering through a dynamic FC. They have used Electrophysiological recordings, EEG data processing, event-related potential and phase locking factor[26-27]. Yuyu Zhang et al.[28] in their research Automatic detection of MW in a simulated driving task with behavioral measures have measured accuracy of 72\% by driving behavior measurements to automatically detect MW state in the driving task. Another group Benjamin Baird et al. [29] and others have worked on decoupled mind by processing EEG data and have found MW disrupts cortical phase-locking to perceptual events. Todd C. Handy[30] and others have worked on MW and selective attention to the external world. The main focus of their work is visual attention, executive function and mental simulation [31-34]. Another group Jonathan Smallwood et al [35] has examined whether the periods of mind wandering are associated with reduced cortical analysis of the external environment. 

Julia W.Y. Kam et al.[34] researched about how the Brain allows us to mentally wander off to another time and place. Kiret Dhindsa et al. [9] have worked on Individualized pattern recognition for detecting MW from EEG during live lectures. They have recorded EEG simultaneously from 15 participants during live lectures and used a data-driven method known as common spatial patterns to discover scalp topologies for each individual that reflects their differences in brain activity when MW versus attending to lectures and achieved an average accuracy of 80-83\%. Jin CY et. Al. [43] have worked on Predicting task-general MW with EEG. They have classified the participants current state by two different paradigm, one is sustained attention to response task (SART) and a visual search task to detect either MW or on task. Qin et al. [44] haveworked on Dissociation of subjectively reported and behaviorally indexed MW by EEG rhythmic activity. By implementing time frequency analysis and means of beamformer source imaging they have found that found subjectively reported MW within the gamma band to be characterized by increased activation in bilateral frontal cortices, supplemental motor area, paracentral cortex and right inferior temporal cortex in comparison to behaviorally indexed MW. Compton RJ ET AL. [45] have worked on the wandering mind oscillates: EEG alpha power is enhanced during moments of MW. During a demanding cognitive task, to find whether episodes of MW increases in EEG alpha power they have used a within-subjects experience-sampling design. 

In the research of Kiret et al.[9] the drawbacks of their work was with only 16 EEG signals they lack the spatial pattern needed for accurate source localization. MW has been mainly detected through two thought-report methods: discrete thought-
probes [49-51] and spontaneous self- reports [49].

A method that have been used is spontaneous self-reports. In this method par- ticipants are requested to specify the moment when they become conscious of MW. From the participant’s side, this method continuously track of MW. This method limits the ability of researchers to maintain consistent evaluation among different participants. We have overcome this problems through our work and analysis.
