% Chapter Template

\chapter{Literature Survey} % Main chapter title

\label{Chapter2} % Change X to a consecutive number; for referencing this chapter elsewhere, use \ref{ChapterX}

\lhead{Chapter 3. \emph{Literature Survey}} % Change X to a consecutive number; this is for the header on each page - perhaps a shortened title
Human mind is not static; it fluctuates over time especially in an attention demanding task. When our brain fluctuates, our mind works in both task-relevant and irrelevant processes but in a less detailed manner.\cite{grandchamp2014oculometric}, We have worked on detecting wandering of our mind using EEG signal in which electrical activity of the brain has been recorded using an electro physiological method. We have also observed the brain waves in EEG signal in the frequency domain. Among various classifiers we have used J48 algorithm to generate a decision tree from our featured data. For classification we have used SVM classifier. By giving a set of trained data example SVM built a model that assigned our data’s in positive or negative MW categories. 

Few researchers also have worked on this field. Julia W. Y. Kam Julia et al. \cite{kam2013mind} have used two experiments, in one experiment they have used traditional measures of performance  and found that both automatic and volitional and  forms of visual–spatial consideration arranging were fundamentally constricted when MW episodes occurred. In the second experiment ERPs have used to examine whether time frames wherein we momentarily weaken the preparing of outer improvement contributions as our considerations float away from the on-going main job or cortical hypersensitivities in headaches reach out to MW.Another group Romain Grandchamp et al. \cite{grandchamp2014oculometric} worked on oculometric variations during MW. They have worked on blink frequency, pupil size and gaze position to check if they were correlated with the occurrence and time course of self-reported MW episodes. 

Another group Jaechoon et al have worked on online education limitations by performing a verification study to see if  mind wandering can be detected using high frequency words to resolve limitations that currently exists. They established a Minimum Learning Judgment System (MLJS) in which vido-based online lecture has been used to detect Mind wandering. 

One more group Benjamin W. Mooney- hamet al.have worked on Mind states by analysising the Neural Bases of alertness and Mind wandering through a dynamic FC. They have used Electrophysiological recordings, EEG data processing, event-related potential and phase locking factor. Yuyu Zhang et al. in their research Auto-detection of MW during a driving task which was simulated, they used behavioral measures have measured accuracy of 72\% .

Another gathering Benjamin Baird et al.  and others have chipped away at decoupled mind by preparing EEG information and have found MW upsets cortical stage locking to perceptual occasions. Todd C. Handy and others have taken a shot at MW and particular consideration regarding the outer world. The fundamental focal point of their work is visual consideration, official capacity, and mental reproduction . Another gathering Jonathan Smallwood et al  has inspected whether the times of psyche meandering are related with the decreased cortical investigation of the outer condition. 

Julia W.Y. Kam et al. looked into how the Brain permits us to intellectually stray to some other time and spot. Kiret Dhindsa et al.  have taken a shot at Individualized example acknowledgment for identifying MW from EEG during live talks. They have recorded EEG at the same time from 15 members during live talks and utilized an information-driven technique known as basic spatial examples to find scalp topologies for every person that mirrors their disparities in cerebrum action when MW as opposed to taking care of talks and accomplished a normal exactness of 80-83\%. Jin CY et. Al.  have taken a shot at Predicting task-general MW with EEG. They have arranged the members ebb and flow state by two distinctive worldviews, one is supported consideration regarding reaction task (SART) and a visual pursuit assignment to identify either MW or on task. Qin et al.  have worked on Dissociation of abstractly revealed and typically ordered MW by EEG cadenced action. By executing time recurrence examination and methods for beamformer source imaging they have discovered that found abstractly revealed MW inside the gamma band to be described by expanded enactment in reciprocal frontal cortices, supplemental engine territory, paracentral cortex and right substandard fleeting cortex in contrast with typically recorded MW. Compton RJ ET AL.  have chipped away at the meandering brain wavers: EEG alpha force is improved during snapshots of MW. During a requesting psychological assignment, to discover whether scenes of MW increments in EEG alpha force they have utilized an inside subjects experience-inspecting plan.

In the examination of Kiret et al. the downsides of their work were with just 16 EEG signals they come up short on the spatial example required for exact source limitation. MW has been essentially identified through two idea report strategies: the discrete idea tests and unconstrained self-reports.

A method that have been used is spontaneous self-reports. In this method participants are requested to specify the moment when they become conscious of MW. From the participant’s side, this method continuously track of MW. There is limitation in  the above methods,that is the researchers ability to maintain evaluation amid different participants consistently. We have overcome this problems through our work and analysis.
